\documentclass{amsart}
\usepackage{graphicx, amsthm}
\usepackage{float}
\usepackage{enumerate}
\restylefloat{figure} 
\usepackage{tikz} 
\newcommand{\Z}{\mathbb{Z}}
\newcommand{\N}{\mathbb{N}}
\newcommand{\R}{\mathbb{R}}
\newcommand{\e}{\mathbf{e}}

\theoremstyle{definition} \newtheorem*{definition}{Definition} 
\theoremstyle{remark} \newtheorem*{ex}{Example} 


\title{Exercise set $6$, \\ MS-A0402, Foundations of Discrete Mathematics}
\begin{document}
\hspace{-1cm}
\maketitle
 
\section*{Explorative exercises}
I recommend that you study the explorative problems before the first lecture of the week.

\subsection*{} Recall that an integer $a\in\Z$ {\em divides} $b\in\Z$ ($a|b$) if there is $n\in\Z$ such that $an=b$.

\subsection*{Problem 1}To practice your understanding of the divisibility definition and of logical symbols, detemine whether the following statements are true or false. (All quantifiers are taken over the integers $\Z$.)
\begin{enumerate}[a)]
\item $\forall a: a|a$.
\item $\forall a: 1|a$.
\item $\forall a: a|1$.
\item $\forall a: 0|a$.
\item $\forall a: a|0$.
\item $\forall a,b: a|b\to b|a$.
\item $\forall a,b,c : \left(a|b\wedge a|c\right)\to a|b+c$.
\item $\forall a,b,c : \left(a|b\wedge b|c\right)\to a|c$.
\item  $\forall a,b :  \left(a|b\wedge b|a\right)\to (a=b\vee a=-b)$.
\end{enumerate}

\subsection*{Problem 2} 
List all the integers that divide 98. Do the same with all numbers that divide 105. What is the {\em greatest common divisor} of $98$ and $105$?

 \subsection*{Problem 3}
The method used in Problem 2 to find the greatest common divisor is very inefficient if the numbers involved are large. Already computing, for example, $\gcd(2331, 2037)$ with this method seems like a disturbingly slow task. We will now find an easier algorithm to do this:
\begin{enumerate}[a)]
\item Show that, if $a,b,n\in \Z$, then the common divisors of $a$ and $b$ are the same as the common divisors of $a$ and $b-na$.
\item Conclude that $$\gcd(2331, 2037)=\gcd(2037, 2331-2037)=\gcd(2037, 294).$$
\item Continue this process, replacing $\gcd(2331, 2037)$ by the greatest common divisors of smaller and smaller numbers.
\item Show that, if $a> 0$, then $\gcd(a,0)=a$.
\item Use this to compute $\gcd(2331, 2037)$.
\end{enumerate}

 \subsection*{Problem 4}
Study the equation $$3x-2y=1.$$ Clearly, it has the integer solution $x=1, y=1$. 
\begin{enumerate}[a)]
\item Can you find more integer solutions? (Hint: if you add  $2$ to the value of $x$, how can you modify  the value of $y$ to get a new solution of the equation?)
\item Can you find {\em all} integer solutions? (And can you prove that there are no others?)
\end{enumerate}
\pagebreak

 \section*{Homework}
The written solutions to the homework problems should be handed in on MyCourses by Monday 11.4., 12:00.  You are allowed and encouraged to discuss the exercises with your fellow students, but everyone should write down their own solutions.

\subsection*{Problem 1}
(10pts) Does the following Diophantine equation 
$$20x + 10y = 65.$$
have solutions $x,y \in \N$? If yes, find all the solutions. If not, justify your answer.


\subsection*{Problem 2}
(10pts) Does the following Diophantine equation 
$$20x+16y=500.$$
have solutions $x,y \in \N$? If yes, find all the solutions. If not, justify your answer.

\subsection*{Problem 3}
(10pts) How many integers less than $22220$ are relatively prime to $22220$?

\subsection*{Problem 4}
(10pts) Compute the last two digits of $2022^{2022}$.







\section*{Additional problems}

These do not need to be returned for marking.

\subsection*{Problem 1}
Show that $7|13^n-6^n$ for all $n\in\Z$.


\subsection*{Problem 2}
Compute 
\begin{enumerate}[a)]\item$3^{19}\mod 13$.
\item $4^{12}\mod 27$.
\item $12^{27}\mod 15$.
\item $146^2\mod 21$
\end{enumerate}

\subsection*{Problem 3}
Are the following statements true or false for arbitrary integers $a$, $b$, and $n$? Prove or find a counterexample.
\begin{enumerate}[a)]
\item If $a\equiv b\mod n$, then $a^2\equiv b^2\mod n$.
\item If $a^2\equiv b^2\mod n$, then $a\equiv b\mod n$.
\end{enumerate}


\subsection*{Problem 4}
Show that $$144|n^8-2n^6+n^4$$ for any integer $n$. (Hint: Factorize!)

\subsection*{Problem 5}
Compute 
\begin{enumerate}[a)]\item$\varphi(200)$.
\item$\varphi(121)$.
\item$\varphi(635)$.
\item$\varphi(1010)$.
\item$\varphi(2021)$.
\end{enumerate}
where $\varphi$ denotes Euler's phi function.

\subsection*{Problem 6}
A seemingly very difficult open question in number theory is whether there are infinitely many ``twin primes'', meaning two consecutive odd numbers that are both primes. Show that there are only finitely many (in fact, only one ``triplet prime''), meaning {\em three} consecutive odd numbers that are all primes.

\subsection*{Exercise 7 (challenging)}
Let $F_n$ be the $n$:th Fibonnacci number. 
\begin{enumerate}[a)]
\item Prove that $\gcd(F_{n}, F_{n-1})=1$.
\item How many steps does Euclid's algorithm take to compute $\gcd(F_n, F_{n-1})$?
\item Show that, if $a,b$ are any numbers with $b\leq a< F_n$, then Euclid's algorithm requires more steps to compute $\gcd(F_n, F_{n-1})$, than to compute $\gcd(a,b)$. (Punchline: The Fibonnacci numbers are the worst possible input for Euclid's algorithm.)
\end{enumerate}



\end{document}