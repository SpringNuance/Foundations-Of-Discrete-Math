\documentclass{amsart}
\usepackage{graphicx, amsthm}
\usepackage{float}
\usepackage{enumerate}
\restylefloat{figure} 
\usepackage{tikz} 
\newcommand{\Z}{\mathbb{Z}}
\newcommand{\N}{\mathbb{N}}
\newcommand{\R}{\mathbb{R}}
\newcommand{\e}{\mathbf{e}}

\theoremstyle{definition} \newtheorem*{definition}{Definition} 
\theoremstyle{remark} \newtheorem*{ex}{Example} 


\title{Exercise set $4$, \\ MS-A0402, Foundations of Discrete Mathematics}
\begin{document}
\hspace{-1cm}
\maketitle
 
\section*{Explorative exercises}
I recommend to study the explorative problems before the first lecture of the week.

\subsection*{Problem 1} 
Recall that there are exactly $m^n$ functions $\{1,\dots ,n\}$ to $\{1,\dots ,m\}$. Why is that?
\begin{enumerate}[a)]
\item How many functions $\{1,\dots ,n\}\to\{1,\dots ,m\}$ are injective? What is required of $n$ and $m$ for injective functions to exist?
\item How many functions from $\{1,\dots ,n\}$ to $\{1,2\}$ are {\em not} surjective?
\item How many functions from $\{1,\dots ,n\}$ to $\{1,2,3\}$ are {\em not} surjective? 
\end{enumerate}

Hint: A non-surjective function must ``miss'' some $i\in\{1,2,3\}$ (meaning that $f(x)\neq i$ for all $x\in \{1,\dots ,n\}$). Count the number of functions that miss $1$, the functions that miss 2, and the functions that miss 3. Did you forget anything?

\subsection*{Problem 2} 
A permutation of $\{1,\dots, n \}$ can be defined in any of two different ways: either as a bijection $\{1,\dots, n \}\to \{1,\dots, n \}$, or as a total order of the set $\{1,\dots, n \}$. Discuss why these two are equivalent.


\subsection*{Problem 3} 
A cycle of length $k$ in a bijection $\sigma: \{1,\dots, n \}\to \{1,\dots, n \}$ is a sequence $a_1, a_2, \cdots a_k$ of distinct elements such that $$\sigma(a_1)=a_2,\,\sigma(a_2)=a_3,\,\cdots  ,\, \sigma(a_{k-1})=a_k,\, \sigma(a_k)=a_1.$$
\begin{enumerate}[a)]
\item  Find all cycles in the permutation $$(\sigma(1), \dots , \sigma(9))=(2,7,5,6,9,3,8,4,1).$$
\item  Is it always true for every permutation $\sigma: \{1,\dots, n \}\to \{1,\dots, n \}$, that every element in $\{1,\dots, n\}$ is in some cycle?
\end{enumerate}

\pagebreak

 \section*{Homework}
The written solutions to the homework problems should be handed in on MyCourses by Monday 28.3 at 12:00. You are allowed and encouraged to discuss the exercises with your fellow students, but everyone should write down their own solutions.

\subsection*{Problem 1}(10pts)
How many integers from $1$ to $60$ are multiples of $2$ or $3$ but not both?

\subsection*{Problem 2}(10pts)
Consider the permutation $$\left(\begin{matrix}1& 2& 3& 4& 5& 6& 7& 8& 9 \\
9&7&3&6&4&2&1&5&8\end{matrix}\right)$$
\begin{enumerate}[a)]
\item Write it as a product of disjoint cycles.
\item Write it as a product of transpositions.
\end{enumerate}

\subsection*{Problem 3}(10pts)
In how many ways can we rearrange the letters in the word 
\begin{center}
“knackered”
\end{center}
\begin{itemize}
\item[(a)] with no restrictions?
\item[(b)] if the first and last letter must be vowels?
\end{itemize}


\subsection*{Problem 4}(10pts)
How many ways are there to tile dominos (with size $2 \times 1$) on a grid of size $2 \times 20$? \\
\textit{(This is a modified job interview question for a Quantitative Researcher position in a London based research firm, $\copyright$\, G-Research} 

\textit{\textbf{Hint} (just a suggestion): Experiment with first $2\times 1$, $2 \times 2$, $2\times 3$, $2\times 4$, etc. sized grids and try to come up with a way to relate
$$a_n = \text{ number of ways to tile a grid of size } 2 \times n$$
to the previous terms $a_{n-1}$ and $a_{n-2}$. Apply this recursive relation then until you reach $a_{20}$.)}



\section*{Additional problems}

These do not need to be returned for marking.

\subsection*{Problem 1}
How many permutations of the 26 letters in the english alphabet contains none of the three words ``cats'', ``snow'', or ``walk''? Here, we say that a sting contains the word `abcd' if the letters $a,b,c, d$ occur next to each other in that order in the string. 

\subsection*{Problem 2}
\begin{enumerate}[a)]
\item $6$ people are first paired up for a dance. Afterwards, they pair up to play a game, where for some reason it is important that the two people in each pair did not dance with each other. In how many ways can this be done?
\item Solve the same problem when there were initially $2n$ participants.
\end{enumerate}

\subsection*{Problem 3}
Write the permutation $( 1 3 6 2 ) ( 2 5 6 4 ) ( 2 3 4 5 )$ \begin{enumerate}[a)]\item as a product of disjoint cycles. \item in two-line notation. \item as a product of transpositions. \end{enumerate}

\subsection*{Problem 4}
Consider the cycles $\rho=( 1 2 3 )$ and $\pi =( 1 2 )$ in $S_3$. \begin{enumerate}[a)]\item Show that
$\rho^3=\pi^2=\iota$. \item Show that $S_3= \{\iota, \rho, \rho^2 \pi, \pi\rho, \pi\rho^2\}$.\end{enumerate}

\subsection*{Problem 5}
Let \begin{align*}a=(-1,-1,-1) ; b=(-1,-1,1) ; c=(-1,1,-1) ; d=(-1,1,1) ; \\ e=(1,-1,-1) ; f=(1,-1,1) ; g=(1,1,-1) ; h=(1,1,1)\end{align*} be the eight corners of a centrally symmetric cube $\square^3$ in three dimensions. Let $\rho$ be rotation by $90^{\circ}$ around the $z$-axis, let $\sigma$ be rotation by $90^{\circ}$ around the $y$-axis, and let $\tau$ be rotation by $90^{\circ}$ around the $x$-axis. Behold my breathtaking drawing skills for an illustration.
\begin{center}
\includegraphics[width=5cm]{cubesymm}
\end{center}
\begin{enumerate}[a)]
\item Write $\rho$, $\sigma$, and $\tau$ as permutations of the set $\{a,b,c,d,e,f,g,h\}$, on two line notation and on cycle notation.
\item Compute $\rho\sigma$, $\sigma\tau$, and $\tau\rho$.
\item How many permutations in $S_8$ can be written as products of $\rho$, $\sigma$, and $\tau$?
\item (challenging) How many symmetries ({\em i.e.} maps $\square^3\to\square^3$ that preserve the distance between corners) are there? Can all of them be written as products of $\rho, \sigma, \tau$? Can you find a symmetry that can not be written as such a product?
\end{enumerate}

\section*{Problem 6}
For what values of $n\in\N$ do there exist sets $A$, $B$ and $C$ such that the following conditions hold: 
\begin{align*}|A|=|B|=|C|=n\\
|A\cap B|=|A\cap C|=|B\cap C|\\
A\cap B\cap C =\emptyset\\
|A\cup B\cup C|=2n
\end{align*}

\subsection*{Problem 7 (challenging)}
\begin{enumerate}[a)]
\item $6$ people celebrate midsummer by dancing in a big circle, each holding hands with the person to their left and to their right. After dinner, they pair up to play a game, where for some reason it is important that the two people in each pair did not hold hands during the dance. In how many ways can this be done?
\item Solve the same problem when there were initially $2n$ participants.
\item Compare this exercise to additional problem 2. Why is this one so much more difficult?
\end{enumerate}

\subsection*{Problem 8 (very challenging)}
Let $X$ and $Y$ be two sets such that there is an injection $f:X\to Y$ and another injection $g:Y\to X$. Show that there is a bijection $$\phi: X\to Y.$$

Note: This proves that our definition of cardinality satisfies the law: $$ \mbox{If }|X|\leq |Y|\leq |X|\mbox{, then }|X|=|Y|.$$
\end{document}  

\end{document}  